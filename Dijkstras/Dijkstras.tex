\documentclass{article}
\usepackage[margin=0.5in]{geometry}
\usepackage{color}
\usepackage{listings}
\lstset{ %
language=C++,                % choose the language of the code
basicstyle=\footnotesize,       % the size of the fonts that are used for the code
numbers=left,                   % where to put the line-numbers
numberstyle=\footnotesize,      % the size of the fonts that are used for the line-numbers
stepnumber=1,                   % the step between two line-numbers. If it is 1 each line will be numbered
numbersep=5pt,                  % how far the line-numbers are from the code
backgroundcolor=\color{white},  % choose the background color. You must add \usepackage{color}
showspaces=false,               % show spaces adding particular underscores
showstringspaces=false,         % underline spaces within strings
showtabs=false,                 % show tabs within strings adding particular underscores
frame=single,           % adds a frame around the code
tabsize=2,          % sets default tabsize to 2 spaces
captionpos=b,           % sets the caption-position to bottom
breaklines=true,        % sets automatic line breaking
breakatwhitespace=false,    % sets if automatic breaks should only happen at whitespace
escapeinside={\%*}{*)}          % if you want to add a comment within your code
}
\usepackage{amsmath}
\usepackage{algorithm}
\usepackage[noend]{algpseudocode}
\begin{document}
	\begin{flushleft}
		Adam Frazee \\*
		CSE 464\\*
		Gregory Gelfond\\*
		11/13/2015\\*
	\end{flushleft}
	\begin{center}
		\LARGE{\textbf{Dijkstra's Algorithm}}
	\end{center}
	\textbf{Formulate The Problem:} Dijkstra's Algorithm is an algorithm used to determine the shortest path between two nodes. In this algorithm we have a graph (G) which is a data structure for holding nodes or (V) vertices, and also has Edges (E). E's have a weight or cost associated with them (W). The source and destination are both nodes in the graph G. V's have a value associated with them is the cost to get to that node from the source node. This means that $V_{Source}$ has a value of 0, and all other V have a value of $\infty$ . At the start of the Algorithm.  \\ \\*
		\large{\textbf{Pseudo Code:}
	\begin{algorithm}
		\caption{Dijkstra}
		\begin{algorithmic}[1]
			\Procedure{Dijkstra}{G,$V_{Source}$,$V_{Destination}$}
			\For{each v $\in$ V}
			\If {v $\equiv V_{source}$}
			\State v.cost $ \leftarrow 0$
			\EndIf
			\State v.cost $ \leftarrow \infty$
			\State  P-Que $\leftarrow $v 
			\EndFor
			
			\While { P-Que has elements}
			\State $V_{cur} \leftarrow$ P-Que.pop
			\If {$V_{cur} \equiv V_{Destination} $}
			\State break
			\For { each neighbor $n$ of $V_{cur}$}
			\State $dist \leftarrow V_{cur}.cost + length(V_{cur},n)$
			\If{$n.cost > dist$}
			\State $dArray[n.cost] \leftarrow dist$
			\State $pArray[n] \leftarrow V_{cur}$
			\EndIf
			\EndFor 
			\EndIf
			\EndWhile
			\State return $dArray,pArray$
			
			\EndProcedure
		\end{algorithmic}
	\end{algorithm}
	\\* \textbf{Proof of Correctness:} 
\end{document}